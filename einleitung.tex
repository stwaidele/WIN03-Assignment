\section{Einleitung}
\label{sec:einleitung}

\subsection{Begründung der Problemstellung}

Die Erwartungen, die an das \buzz{Internet of Things} (IoT) als Technologie gestellt werden, sind im Moment so groß wie bei keiner anderen neuen Technologie.\footnote{\cite{ghc14}}. Gartner betrachtet das IoT als Megatrend und maßgeblich treibende Kraft hinter einzelnen Technologien, die gerade entwickelt werden und deren volles Potential erst in den nächsten Jahren ersichtlich werden wird\footnote{\cite{gartner2014}, Figure 1}.\\
Die Plazierung von \buzz{QR–Codes}, \buzz{RFID–Chips} oder \buzz{NFC–Tags} erlaubt schon heute Tracking von  Gegenstände und Personen in deutlich größerer Anzahl und höherer Genauigkeit als konventionelle Methoden.

Bein \buzz{Ubiquitous Computing} (UC) oder \buzz{Wearables} hingegen wird tatsächlich Rechenleistung in alltägliche Gegenstände wie Armbanduhren, Laufschuhe oder Kühlschränke verpackt. Auch hier erfolgt eine Erfassung von lokalen Daten. Diese werden dann aber mit einem erheblich größeren Anteil von herkömmlichen Daten aus dem Internet ergänzt\footnote{Etwa mit einer Rezeptdatenbank oder einer Landkarte}.

Sowohl die vor Ort generierte Datenmenge als auch die Anzahl der verarbeitenden Instanzen wird in den nächsten Jahren wohl deutlich zunehmen.

\subsection{Ziele dieser Arbeit}

\textbf{Ziel dieser Arbeit ist es, Potentiale und Herausforderungen der Datenverarbeitung und Weiterleitung im Internet of Things zu identifizieren sowie entsprechende Lösungsansätze zu skizzieren.}

Hierbei soll das Hauptaugenmerk auf firmeninterne Prozesse bzw. lokal begrenzte Anwendungen gelegt werden. Hierbei kann auch vom \buzz{Intranet of Things} gesprochen werden.

Hierzu werden zunächst im Kapitel~\myref{sec:grundlagen} die für diese Arbeit relevanten Begriffe und Konzepte definiert, bevor im Kapitel~\myref{sec:technologien} die momentan verfügbaren Technologien genannt und erklärt werden. 

Darauf aufbauend werden im Kapitel~\myref{sec:herausforderungen} die technischen Probleme identifiziert, und Lösungsansätze skizziert. Im Kapitel~\myref{sec:potentiale} werden schließlich einige mit dem IoT erwachsenden Möglichkeiten beschrieben. 

\todo{An die tatsächliche Arbeit anpassen}

\subsection{Abgrenzung}

\todo{Wahrscheinlich Datenschutz außer Betracht lassen.}

\todo{An die tatsächliche Arbeit anpassen}
