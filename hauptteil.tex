\section{Herausforderungen}
\label{sec:herausforderungen}

\subsection{Dezentrale Datenverarbeitung, Data Warehouse, Big Data}
\subsection{Konfiguration von Rechten und Prozessen}
\subsection{Problemspezifische Apps zur verteilten Datenverarbeitung}
\subsection{Wem gehören die Rechte?}

Vorschlag: Meine Daten gehören mir. Die Datensammlung gehört dem Datensammler.
Siehe auch Kapitel~\myref{sec:autogathering}

\section{Potentiale}
\label{sec:potentiale}

\subsection{Appliances++}
\subsection{Lagerverwaltung++}
\subsection{Customerservice++}

z.B. Gast morgens an der Rezeption. Das Infoterminal zeigt die Wettervorhersage für den heute gebuchten Ausflug bzw. für den Heimweg an.
z.B. Gast abends im Zimmer. Der Hotelfernseher zeigt Dokumentation über die unterwegs besuchte Sehenswürdigkeit (Basierend auf tatsächlicher, nicht gebuchter Route).

\subsection{Automatische Datenherausgabe}
\label{sec:autogathering}

Eine automatisierte Sammlung aller Daten, die über ein \buzz{Thing} gesammelt wurden ermöglicht nicht nur den Datensammlern, sondern auch den Individuen, diese (eigenen) Daten zu nutzen und auszuwerten.

\subsection{Der Mehrwert von „Mehr“ und „Schneller“}

Viele Entwicklung der Informationstechnologie basieren auf gundlegenden Techniken, die schon lange Bekannt sind. z.B: Digitaler Surroudsound basiert auf den gleichen Grundlagen wie 8–Bit Musik aus dem C64. GIF–Animationen basieren auf dem gleichen Prinzip wie Malen nach Zahlen und Daumenkino. Nur eben „mehr“ bzw. „schneller“. 