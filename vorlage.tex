% Vorgaben Assignment aus Studienheft SQL03
% Formatvorgaben fuer den Text
% Umfang: 8 - 10 Seiten (inkl. Abbildungen und Tabellen, aber ohne Deckblatt, % Gliederung und Literaturverzeichnis, Eidesstattliche Erklaerung)
% Zeilenabstand: 1,5
% Schriftart: frei
% Schriftgrad: 12 pt
% Variablen, physikalische Groessen und Funktionszeichen werden kursiv gedruckt.
% Korrekturrand: links: 4,5 cm, rechts 2,0 cm, oben und unten jeweils 3,0 cm
% Deckblatt: (Adresse, AKAD-E-Mail-Adresse, Immatrikulationsnummer, Modul-
% bezeichnung, Thema, Datum, Felder für Korrektor)
% Gliederung (1 Seite)
% Literaturverzeichnis (3 - 5 Literaturquellen  z. B. Lehrbuecher, aktuelle Fachartikel recherchieren)
% Eidesstattliche Erklaerung (unterschrieben und fest eingebunden)
% Bearbeitungsdauer: 2 Monate


\documentclass[a4paper,12pt]{article}
\usepackage[ngerman]{babel}
\usepackage[nottoc]{tocbibind} % Anzeigen des Literaturverzeichnisses im TOC
\usepackage{epsfig}
\usepackage{times}
\usepackage{supertabular}
\usepackage{wrapfig}
\usepackage{multirow}
\usepackage[onehalfspacing]{setspace}
\usepackage{listings}
\usepackage{mathptmx}
\usepackage{geometry}
\usepackage{helvet}
\usepackage{courier}
\usepackage{setspace}
\usepackage{textcomp}
\usepackage[T1]{fontenc}
\usepackage[utf8]{inputenc}
\usepackage{fancyhdr}
\usepackage{float} % Notwendig fuer figure[h]
\usepackage[printonlyused]{acronym}

\newif\iflistoffigures
\newif\iflistoftables
\newif\ifacronym

%Titel
\newcommand*{\Titel}{Intranet der Dinge} 

%Betreff
\newcommand*{\Betreff}{Assignment} 

%Betreuer
\newcommand*{\Betreuer}{Prof. J. Anton Illig} 

%Vor- und Nachname
\newcommand*{\Name}{Stefan Waidele}

%Straße und Hausnummer
\newcommand*{\Strasse}{Ensisheimer Straße 2} 

%Plz und Ort
\newcommand*{\PlzOrt}{79395 Neuenburg am Rhein} 

%Immatrikulationsnummer
\newcommand*{\Immatrikulationsnummer}{102 81 71}

%Email 
\newcommand*{\Email}{Stefan@Waidele.info} 


% Verzeichnisse (Wenn nicht benötigt, Zeile mit % auskommentieren oder löschen

%% Abbildungsverzeichnis 
%\listoffigurestrue
%% Tabellenverzeichnis
%\listoftablestrue
%% Abkürzungsverzeichnis
%\acronymtrue

% Wittwen und Waisen verhindern
\clubpenalty10000
\widowpenalty10000
\displaywidowpenalty=10000

\usepackage[flushmargin,hang,ragged]{footmisc}
\usepackage{lmodern} %Type1-Schriftart für nicht-englische Texte
\usepackage{fancyhdr}

\usepackage[
	pdftitle={\Titel},
	pdfsubject={WIN03 -- Innovative Themen der Wirtschaftsinformatik},
	pdfauthor={Stefan Waidele},
	pdfkeywords={akad, win03, assignment, wirtschaftsinformatik}
	hyperfootnotes=false,
	colorlinks=true,
	linkcolor=black,
	urlcolor=black,
	citecolor=black
]{hyperref}

%\renewcommand{\familydefault}{\rmdefault}
\renewcommand{\bflabel}[1]{\normalfont{\normalsize{#1}}\hfill}

%% Definition for Codeschnipsel im Fließtext
\newcommand{\code}{\texttt}
\newcommand{\buzz}{\textit}
\newcommand{\todo}[1]{\fbox{\parbox{\textwidth}{\textbf{To do:} #1}}}
%\newcommand{\myref}[1]{„\ref{#1}~\nameref{#1}“}
\newcommand{\myref}[1]{\textit{\ref{#1}~\nameref{#1}}}

%% Für Codeblöcke mit Syntax-Highlighting
%% http://www.ctan.org/tex-archive/macros/latex/contrib/minted/
\usepackage{minted}
\definecolor{bg}{rgb}{0.95,0.95,0.95}


\makeatother

\geometry{a4paper, left=45mm, right=20mm, top=30mm, bottom=30mm}
\pagenumbering{roman}

\begin{document}
	
\parskip=1em
\parindent=0cm

\include{deckblatt}

\normalsize

\begin{spacing}{1.0} % Verzeichnisse werden mit einzeiligem Abstand gesetzt
\parskip=0em
\newpage

% Inhaltsverzeichnis
\setcounter{tocdepth}{2}
\tableofcontents 
\newpage

% Abbildungsverzeichnis
\iflistoffigures
\listoffigures 
\newpage
\fi

% Tabellenverzeichnis
\iflistoftables
\listoftables 
\newpage
\fi

% Abkürzungsverzeichnis
\ifacronym
\include{abkuerzungen}
\fi

\parskip=1em
\end{spacing} 

\clearpage

\newcounter{romanPagenumber} 
\setcounter{romanPagenumber}{\value{page}} % Roemische Seitenanzahl speichern.


\pagestyle{fancy}
\fancyhead{}
\fancyhead[LO,RE]{\textsc{\Titel}}
\fancyhead[RO,LE]{\thepage}
\fancyfoot[CO,CE]{}
\setlength{\headheight}{15pt}

% \nocite{*} 

\pagenumbering{arabic}

\begin{spacing}{1.5} % Zeilenabstand: 1,5 fuer den Textteil

\section{Einleitung}
\label{sec:einleitung}

\subsection{Begründung der Problemstellung}

Technologien rund um das Schlagwort \buzz{Big Data}\index{Big Data} sind laut Gartner die großen Triebfedern in der Informationstechnologie\footnote{vgl. \cite{gartner2014}}. Durch Entwicklungen wie \buzz{Internet of Things}, \buzz{Ubiquitous Computing}\index{Ubiquitous Computing} und \buzz{Life Tracking} werden die in naher Zukunft die generierte Datenmenge als auch die Anzahl der verarbeitenden Instanzen in den nächsten Jahren deutlich zunehmen.

Die steigende Menge der Daten macht eine systematische Aufbereitung der anfallenden Daten hin zu repräsentiertem Wissen möglich und notwendig. Das semantische Web, oder auch \buzz{Web 3.0}\index{Web!3.0}, verspricht Struktur in die Datenmenge zu bringen. Die Erwartungen, aber auch die Befürchtungen gegenüber den entsprechenden technischen Fortschritten sind immens.

\subsection{Ziele dieser Arbeit}

\textbf{Ziel dieser Arbeit ist es, die momentanen Entwicklungen der Datenbeständen hin zum semantischen Web und deren Auswirkungen auf Wirtschaft und Gesellschaft mit den durch die Industrielle Nutzung des Öls im 20. Jahrhundert zu vergleichen.}

Hierzu werden zunächst im Kapitel~\myref{sec:grundlagen} die für diese Arbeit relevanten Begriffe und Konzepte definiert, bevor im Kapitel~\myref{sec:hauptteil} die Auswirkungen der Technologien auf Wirtschaft und Gesellschaft beschrieben und in Abschnitt~\myref{vergleich} mit denen des Öls verglichen werden.

\subsection{Abgrenzung}

Der Augenmerk dieser Arbeit liegt auf den behandelten Konzepten und Technologien der Informationstechnologie. Die Entsprechungen in der Ölindustrie werden nicht in der gleichen Tiefe erörtert und belegt werden, sondern beschränken sich auf generelle Aussagen und Ansichten.

Die Themenbereiche \buzz{Internet of Things}, \buzz{Ubiquitous Computing}\index{Ubiquitous Computing} und \buzz{Life Tracking} werden in dieser Arbeit unter dem Begriff \buzz{Ubiquitous Computing}\index{Ubiquitous Computing|textbf}\footnote{ubiquitous: Englisch für „allgegenwärtig“} zusammengefasst, auch wenn damit unterschiedliche Techniken, Geräte und Anwendungsfälle beschrieben werden können. Ebenfalls werden diese Gebiete als stetig wachsende Datenquellen angesehen, ohne dass sie genauer untersucht werden\footnote{Siehe hierzu auch Abschnitt~\myref{ausblick}}.
\section{Grundlagen}
\label{sec:grundlagen}

\subsection{Veredlungsprozess von Öl}

Rohöl: Finden / Fördern\\
Raffinerie: Sammeln / Aufbereiten\\
Anwendung: Verschiedene Ölprodukte, aber auch Kunststoffe, Kaugummi, etc.

\subsection{Auswirkungen des Öls auf die Wirtschaft des 20. Jahrhunderts}

Für Unternehmen und Verbraucher

\subsection{Definition: Daten, Informartion, Wissen}

\subsection{Semantisches Markup}

Beschreibung der Daten außerhalb der Datei

\subsection{Metadaten}

Beschreibung der Daten in der Datei (z.B. EXIF--Tags)

\subsection{Automatische Informationsgewinnung}

z.B. Gesichtserkennung

\subsection{Automatische Wissensgewinnung}

Durch Verknüpfung von Informationen kann Wissen generiert werden:\\
Person A ist auf einem Bild zusammen mit Person B zu sehen (Gesichtserkennung). Die Geoinformation und Uhrzeit (EXIF--Tags) zeigen, dass das Foto auf einer Veranstaltung aufgenommen wurde, das zu diesem Zeitpunkt an diesem Ort statt fand (Semantisches Markup der Veranstaltung).

Daraus kann auf politische Gesinnung oder Trinkfreudigkeit von Person A geschlossen werden.
\section{Hauptteil}
\label{sec:hauptteil}

\subsection{Auswirkungen des Semantischen Webs auf die Wirtschaft}

\subsection{Auswirkungen des Semantischen Webs auf die Gesellschaft}

Schon heute drehen sich die meisten der regelmäßig ausgeführtem Aktivitäten im Internet um \buzz{Informationen} oder gar \buzz{Wissen}, nicht um \buzz{Daten}. So sind 56\% der im Digitalindex 2014 befragten Deutschen überzeugt, im Internet die automatisch die aktuellsten Informationen zu finden, 60\% sucht benötigte Informationen zuerst im Netz\footnote{vgl. \cite{d21}, Seite 6}. Die Erwartungen bzgl. Aktualität und Informationsgehalt an das \ac{WWW} sind also sehr hoch. 

\begin{figure}[H]
\begin{center}
\includegraphics[width=0.67\textwidth]{inetnutzung.jpg}
\caption[Internetnutzung in Deutschland 2014]{Internetnutzung in Deutschland 2014\protect\footnotemark}
\label{pic:inetnutzung}
\end{center}
\end{figure}
\footnotetext{\cite{d21}, Seite 37}


Unter den Top 10 der regelmäßig durchgeführten Tätigkeiten der Befragten im Web finden sich die informationsorientierten Tätigkeiten „nach Inhalten/Informationen suchen“ auf Platz eins, „über aktuelle Ereignisse des Wohnorts informieren“ auf Platz 6 und die Erfassung eigener Daten auf Platz 10\footnote{vgl. \cite{d21}, Seite 37}. Weitere Tätigkeiten wie „Soziale Netzwerke nutzen“ (Platz 7) bzw. „Online--Videos ansehen“ (Platz 3) sowie „Online--Shopping“ (Platz 2) nutzen Internetangebote, die per Design sehr gut mit Taxonomien, Meta--Daten und Verknüpfungen ausgestattet sind.

Neben dem offensichtlichen Nutzen des \ac{WWW} bringt die Entwicklung des Webs auch negative Auswirkungen auf die Gesellschaft. 
Dies äußert sich mit der Besorgnis von 60\% der Nutzer über die im Internet möglicherweise verfügbaren persönlichen Daten\footnote{vgl. \cite{d21}, Seite 6}. 

Man kann davon ausgehen, dass mit weiter wachsenden Datenmengen, aber auch durch entsprechenden Wachstum an generierten und erfassten Informationen und Wissen sowohl die positiven Erwartungen als auch die Befürchtungen und Ängste in der Bevölkerung zunehmen werden.


\subsection{Vergleich der Auswirkungen mit denen des Öls}

\subsection{}

\section{Fazit \& Ausblick}

\subsection{Fazit}

In der Analogie zu Erdöl lassen sich die \buzz{Daten} wie in der These beschrieben mit Rohöl vergleichen. Die Techniken rund um \buzz{Big Data} gemeinsam mit den datenbankbezogenen Schichten des \buzz{semantischen Webs} entsprechen somit den Ölspeichern, welche den Rohstoff bereit halten. Die Weiterverarbeitung erfolgt dann in den logischen Schichten des \buzz{Web 3.0}, welche somit am ehesten mit den Raffinerien vergleichbar sind, die aus den klebrigen schwarzen Rohöl der Daten die verschiedenen Informations-- und Wissensprodukte erzeugt. Ob diese dann mit Schnierstoffen, schwerem Schiffsdiesel, hocheffizientem Kerosin oder Kunststoffen vergleichbar sind hängt sowohl von den Ausgangsdaten, aber auch wie beim Öl von den Zielsetzung und Anforderungen ab.

Datenschutz ist in diesem Bild vergleichbar mit Umweltschutz, der auf allen Ebenen dafür kämpft, dass die negativen Auswirkungen der neuen Technologie gemindert bzw. eliminiert werden. Wo beim Öl Strände, Seevögel, die Atmosphäre und das Klima geschützt werden, sind es  im Umfeld der Informationstechnologie Bürger--, Grund-- und Persönlichkeitsrechte, die vor übermäßigem und falschem Einsatz der Technologie geschützt werden sollen.

\subsection{Ausblick}

Nach dieser Einordnung des \buzz{Web 3.0} bietet sich für weitere Arbeiten die genauere Untersuchung der einzelnen Aspekte an, die hier zum großen Teil nur genannt werden konnten. Speziell die automatische Informationserkennung durch Mustererkennung in Texten, Grafiken und Videos und die daraus entstehenden Möglichkeiten sind eine nähere Betrachtung wert. Ebenso die höheren Ebenen des Web 3.0 Stacks, in denen es ebenfalls darum geht, Zusammenhänge rechnergestützt zu erkennen. Durch die Automatisierung der Informationsgewinnung wird diese skalierbar, so wie es die Datenspeicherung heute schon ist. Dies wird die in dieser Arbeit beschriebenen Entwicklungen nochmals deutlich beschleunigen und in ihrer Tragweite vergrößern. Daher empfiehlt sich auch eine gründliche Untersuchung der gesellschaftlichen und politischen Aspekte des Semantischen Webs.
\include{anhang}

\end{spacing}

\clearpage

% Literaturverzeichniss - Ab hier wieder Roemische Seitenzahlen
\pagestyle{plain}
\pagenumbering{roman}
\setcounter{page}{\theromanPagenumber}
%\addcontentsline{toc}{section}{Literatur-- und Quellenverzeichnis}
\renewcommand{\refname}{Literatur-- und Quellenverzeichnis}
\bibliographystyle{apalike}
\bibliography{literatur}
\onehalfspacing
\clearpage

\pagestyle{empty} 
\thispagestyle{empty}

\begin{center}
{\Large Eidesstattliche Erklärung}
\vspace*{4cm}\end{center}
\noindent
Ich versichere, dass ich das beiliegende Assignment selbstständig verfasst, keine anderen als die angegebenen Quellen und Hilfsmittel benutzt sowie alle wörtlich oder sinngemäß übernommenen Stellen in der Arbeit gekennzeichnet habe. 
\vspace{3cm}

\hspace{-0.8cm}
\rule[0.5ex]{6.5cm}{1pt}
\hspace{1.3cm}
\rule[0.5ex]{6.5cm}{1pt}
\\(Datum, Ort)
\hspace{6.3cm}
(Unterschrift)

\clearpage
\end{document}

